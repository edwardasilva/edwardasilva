\documentclass[11pt]{article}
\usepackage{geometry}
\geometry{letterpaper,top=0.5in,bottom=0.5in,left=0.5in,right=0.5in}

\usepackage{XCharter}
\usepackage[T1]{fontenc}
\usepackage[utf8]{inputenc}
\usepackage{enumitem}
\usepackage[hidelinks]{hyperref}
\usepackage{bookmark}
\usepackage{titlesec}
\raggedright
\pagestyle{empty}

\input{glyphtounicode}
\pdfgentounicode=1

% Document metadata for PDF viewers
\title{Edward Silva - Resume}
\author{Edward Silva}

% PDF viewer section encoding
\hypersetup{
    pdftitle={Edward Silva - Resume},
    pdfauthor={Edward Silva},
    pdfsubject={Resume},
    pdfkeywords={Edward Silva, C++, MATLAB, Java, Python, Verilog, RISC-V Assembly, Bash, Colorado School of Mines, BS in Electrical Engineering, Controls & Signal Processing, Minor in Computer Science, Software and Algorithm Design},
    colorlinks=false,
    linkcolor=black,
    urlcolor=black,
    citecolor=black,
    filecolor=black
}

\titleformat{\section}{\bfseries\large}{}{0pt}{}[\vspace{1pt}\titlerule\vspace{-10pt}]

\renewcommand\labelitemi{$\vcenter{\hbox{\small$\bullet$}}$}
\setlist[itemize]{itemsep=-2pt, leftmargin=12pt}

\begin{document}

% Document metadata (invisible for ATS/AI parsing)
\thispagestyle{empty}

% Main document bookmark
\pdfbookmark[0]{Edward Silva - Resume}{main}

\centerline{\huge Edward Silva}
\vspace{5pt}

\centerline{
\href{https://www.linkedin.com/in/edwardasilva/}{Linkedin.com/in/edwardasilva}
| \href{https://easilva.com}{easilva.com}
| \href{mailto:contact@easilva.com}{contact@easilva.com} 
| \href{tel:702 720-7735}{(702) 720-7735}
}

\vspace{-14pt}
\section*{Experience}
\pdfbookmark[1]{Experience}{experience}
\vspace{5pt}

\textbf{Software Engineering Intern, }{Kratos Defense} -- Colorado Springs, CO \hfill June -- August 2025 \\
\vspace{-5pt}
\begin{itemize}
  \item Optimized legacy DSP algorithms in C++ through code refactoring and performance analysis, achieving 1.6x execution speedup and reducing computational overhead for real-time signal processing applications.
  \item Developed and implemented SIMD-optimized mathematical algorithms using vectorized operations, enabling parallel data processing and improving system throughput for multi-channel signal analysis.
  \item Designed and deployed a comprehensive logging framework with configurable severity levels and error tracking, reducing debugging time and improving system maintainability for development teams.
  \item Contributed to agile development practices using Jira for sprint planning, task tracking, and project management, enhancing team collaboration and delivery efficiency.
\end{itemize}

\textbf{Co-op Intern, Electrical Design, }{Jordan and Skala Engineers} -- Denver, CO \hfill January -- June 2025 \\
\vspace{-5pt}
\begin{itemize}
  \item Contributed to electrical design of 20+ multi-unit residential and specialty building developments, spanning initial takeoffs, layout design, riser diagrams, NEC verification, and QC review.
  \item Developed proficiency in Autodesk Revit and MEP AutoCAD, strategically placing electrical receptacles, lighting, and circuits to ensure NEC compliance and practical, user-centered functionality.
  \item Performed circuit loading and voltage drop calculations, balancing panel schedules and selecting appropriate breakers to ensure safety, reliability, and adherence to regulatory standards.
  \item Utilized existing automation between Revit/CAD layouts and Excel tracking sheets to streamline design documentation processes and reduce manual errors.
  \item Collaborated closely with supervisors and cross-disciplinary teams (Mechanical, Plumbing), documenting client interactions and team meetings to improve project coordination and team efficiency.
\end{itemize}

\vspace{-14pt}
\section*{Projects}
\pdfbookmark[1]{Projects}{projects}
\vspace{5pt}
\textbf{Dual-Axis Solar Tracker Robot}, Arduino, Raspberry Pi, C++, \href{https://github.com/edwardasilva/SolarPanelProject}{Github} \hfill August -- October 2024
\vspace{-5pt}
\begin{itemize}
  \item Designed and built a dual-axis solar tracking prototype using Arduino-controlled servos and photoresistor-based voltage divider circuits to maximize solar exposure.
  \item Wrote a custom tracking algorithm from scratch to identify the brightest point in the sky through light intensity sampling, enabling precise pitch and yaw adjustments.
  \item Utilized a Raspberry Pi as the system's central controller, handling logic flow and interfacing with the Arduino to execute real-time motor positioning.
  \item Conducted iterative indoor testing to calibrate sensitivity and response thresholds under varying lighting conditions, improving tracking accuracy and stability.
\end{itemize}

\vspace{-18pt}
\section*{Education}
\pdfbookmark[1]{Education}{education}
\vspace{1pt}

\textbf{\href{https://www.mines.edu/}{Colorado School of Mines}}, \textbf{GPA:} 3.44  \hfill Expected May 2026\\
\textbf{\href{https://electrical.mines.edu/undergraduate-program/}{BS in Electrical Engineering}} -- Controls \& Signal Processing  \\
\href{https://cs.mines.edu/csmines-minors-and-areas-of-special-interest/}{Minor in Computer Science} -- Software and Algorithm Design\\
\textbf{Courses:} Advanced Control Systems, Signals \& Systems, Embedded Systems, Software Engineering
\textbf{Certifications:} Microsoft Technical Associate (MTA): Python \& Java Programming

\vspace{-14pt}
\section*{Skills}
\pdfbookmark[1]{Skills}{skills}
\vspace{5pt}

\textbf{Programming Languages:} C++, MATLAB, Java, Python, Verilog, RISC-V Assembly, Bash \\
\textbf{Technology:} SSH, Linux OS (Ubuntu), Raspberry Pi, Arduino \\
\textbf{Software:} Autodesk Revit, MEP AutoCAD, VS Code, GitHub \\

\end{document}