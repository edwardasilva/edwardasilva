
\documentclass[11pt]{article}
\usepackage{geometry}
\geometry{letterpaper,top=0.5in,bottom=0.5in,left=0.5in,right=0.5in}

\usepackage{XCharter}
\usepackage[T1]{fontenc}
\usepackage[utf8]{inputenc}
\usepackage{enumitem}
\usepackage[hidelinks]{hyperref}
\usepackage{bookmark}
\usepackage{titlesec}
\usepackage[protrusion=true,expansion=true]{microtype}
\raggedright
\pagestyle{empty}

\input{glyphtounicode}
\pdfgentounicode=1

\title{Edward Silva - Resume}
\author{Edward Silva}

\hypersetup{
    pdftitle={Edward Silva - Resume},
    pdfauthor={Edward Silva},
    pdfsubject={Resume},
    pdfkeywords={Edward Silva, C++, Python, MATLAB, Java, Verilog, RISC-V Assembly, Bash, Colorado School of Mines, BS Electrical Engineering, Controls and Signal Processing, US Citizen},
    colorlinks = false,
    linkcolor = black,
    urlcolor = black,
    citecolor = black,
    filecolor = black
}

% Section styling and spacing
\titleformat{\section}{\bfseries\large}{}{0pt}{}[\vspace{2pt}\titlerule]
\titlespacing*{\section}{0pt}{0.70\baselineskip}{0.70\baselineskip}

% Consistent list spacing across document
\renewcommand\labelitemi{$\vcenter{\hbox{\small$\bullet$}}$}
\setlist[itemize]{leftmargin=12pt,topsep=3pt,itemsep=3pt,parsep=0pt,partopsep=0pt}

\begin{document}

\thispagestyle{empty}
\pdfbookmark[0]{Edward Silva - Resume}{main}

\centerline{\huge Edward Silva}
\vspace{3pt}

\centerline{\href{https://www.linkedin.com/in/edwardasilva}{linkedin.com/in/edwardasilva} | \href{https://easilva.com}{easilva.com} | \href{mailto:contact@easilva.com}{contact@easilva.com} | \href{tel:7027207735}{(702) 720-7735}}

\section*{Education}
\pdfbookmark[1]{Education}{education}

\textbf{Colorado School of Mines} \hfill Expected May 2026 \\
\textbf{BS Electrical Engineering} , Minor in Computer Science \\



\section*{Skills}
\pdfbookmark[1]{Skills}{skills}

\textbf{Programming Languages:} C++, Python, MATLAB, Java, Verilog, RISC-V Assembly, Bash \\
\textbf{Hardware:} Arduino, Raspberry Pi, Digital Circuits, Embedded Systems, Microcontrollers, Circuit Design \\
\textbf{Software:} VS Code, Git/GitHub, Linux, Simulink, Autodesk Revit, MEP AutoCAD, SSH, LaTeX \\

\section*{Experience}
\pdfbookmark[1]{Experience}{experience}

\textbf{Software Engineering Intern, }{Kratos Defense} -- Colorado Springs, CO \hfill June - August 2025 \\
\begin{itemize}
  \item Achieved 1.6x execution speedup by optimizing legacy DSP algorithms in C++ through code refactoring and performance analysis, reducing computational overhead for real-time signal processing applications.
  \item Improved system throughput by developing and implementing SIMD-optimized mathematical algorithms using vectorized operations for parallel data processing.
  \item Researched and demonstrated an improved approach to coding a FIR filter, presenting positive findings and performance gains to the team for adoption in future projects.
  \item Reduced debugging time and improved system maintainability for development teams by designing and deploying a comprehensive logging framework with configurable severity levels and error tracking.
\end{itemize}

\section*{Projects}
\pdfbookmark[1]{Projects}{projects}

\textbf{Autonomous Path Following Robot}, Arduino, Raspberry Pi, Python, C++  \hfill August 2025 - Present
\begin{itemize}
  \item Developing the computer vision system for a semester-long robotics project, working within a four-person team split between vision and controls
  \item Created a real time object detection program in Python using OpenCV that identifies target shapes from live video streams with bounding boxes and masks, surpassing the original single image requirement
  \item Building a communication interface between the Raspberry Pi and Arduino to exchange control and sensor data, enabling integration of perception with motion control
  \item Supporting integration with a PID control system to achieve path following and autonomous navigation
\end{itemize}

\vspace{4pt}

\textbf{Crane Gantry Controller}, MATLAB, Simulink  \hfill August - October 2025
\begin{itemize}
  \item Modeled and linearized a nonlinear gantry crane system in MATLAB and Simulink using small-angle approximations, deriving the transfer function and validating results against experimental data
  \item Identified and tuned the crane's cable length parameter to match empirical position and velocity data with accuracy to three decimal points
  \item Designed and implemented a discrete-time feedback controller that met overshoot, settling time, and velocity constraints while maximizing gain and phase margins
  \item Analyzed closed-loop system performance and stability margins through time-domain and frequency-domain plots to confirm robust controller behavior for both linear and nonlinear models
\end{itemize}

\section*{Extracurriculars}
\pdfbookmark[1]{Extracurriculars}{extracurriculars}
\small Astronomy Club, IEEE, ACM, Hiking

\end{document}
